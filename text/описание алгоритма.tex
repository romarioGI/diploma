\section{Алгоритм Тарского}

Прежде всего определим область математики, истинность утверждений которой должен проверять алгоритм.
Затем зададим язык, на котором записываются эти утверждения.
И, наконец, опишем алгоритм, который по формуле описанного языка строит эквивалентную бескванторную формулу.

\subsection{Элементарная алгебра}

Под элементарной алгеброй понимается та часть общей теории действительных чисел, в которой используются переменные, представляющие собой действительные числа, и константы для всех рациональных чисел, также в которой заданы арифметические операции, такие как <<сложение>> и <<умножение>>, и отношения сравнения действительных чисел~--- <<меньше>>, <<больше>> и <<равно>>.
То есть рассматриваются системы алгебраических уравнений и неравенств.

Заметим, что используя декартову систему координат, большую часть всех задач геометрии можно сформулировать как задачи элементарной алгебры. 

Например, теорема о пересечении высот треугольника, которая утверждает, что три высоты невырожденного треугольника пересекаются в одной точке, равносильна утверждению: для любых трех точек $A(x_1, y_1)$, $B(x_2, y_2)$ и $C(x_3, y_3)$, не лежащих на одной прямой, существует точка $D(x_4, y_4)$ такая, что $\overrightarrow{AD} \perp \overrightarrow{BC}$, $\overrightarrow{BD} \perp \overrightarrow{AC}$ и $\overrightarrow{CD} \perp \overrightarrow{AB}$. 

Или иначе говоря, если $\overrightarrow{AB} \land \overrightarrow{AC} \neq 0$, то система
\begin{equation*}
    \begin{cases}
        \left(\overrightarrow{AD}, \overrightarrow{BC}\right) = 0 \\
        \left(\overrightarrow{BD}, \overrightarrow{AC}\right) = 0 \\
        \left(\overrightarrow{CD}, \overrightarrow{AB}\right) = 0
    \end{cases}
\end{equation*}
имеет решение относительно переменных $x_4$, $y_4$, где $* \land *$~--- псевдоскалярное произведение векторов, $(*, *)$~--- скалярное произведение векторов.

Продолжая эти рассуждения, можно определить формальный язык элементарной алгебры.

\subsection{Язык элементарной алгебры}

Язык элементарной алгебры~--- это язык логики первого порядка с сигнатурой
\begin{equation*}
    \tau = \langle \, \mathbb{Q},\, F,\, P,\, \theta,\, \phi \, \rangle,
\end{equation*}
где
\begin{itemize}
	\item $\mathbb{Q}$~--- множество рациональных чисел, которое является множеством индивидных констант;
	\item $F =  \left\{+,\, \cdot\right\}$~--- множество функциональных символов;
	\item $P = \left\{ <,\, >,\, = \right\}$~--- множество предикатных символов;
	\item $\theta : F \rightarrow \mathbb{N}$ такое, что $\theta(+) = 2$ и $\theta(\cdot) = 2$;
	\item $\phi : P \rightarrow \mathbb{N}$ такое, что $\phi(<) = 2$, $\phi(>) = 2$ и $\phi(=) = 2$. 
\end{itemize}

Из определения отображений $\theta$ и $\phi$ видно, что все $f \in F$ являются двухместными функциональными символами, а все $p \in P$ являются двухместными предикатными символами.

Основное множество интерпретации языка $L_\tau $ совпадает с множеством действительных чисел $\mathbb{R}$.

Отображение множества индивидных констант в основное множество определяется естественным образом, так как $\mathbb{Q} \subset \mathbb{R}$.Функциональные символы $+$ и $\cdot$ отображаются в сложение и умножение в поле $\mathbb{R}$ соответственно. И предикатные символы $<$, $>$ и $=$ отображаются естественным образом в операции сравнения в $\mathbb{R}$.

Заметим, что множество констант ограничено рациональными числами лишь потому, что компьютер может быстро работать с ними без потери точности, что нельзя сказать про действительные числа.


Таким образом, теорему о пересечении высот на языке элементарной алгебры можно записать следующей формулой:
\begin{gather*}
    (\forall x_1)(\forall y_1)(\forall x_2)(\forall y_2)(\forall x_3)(\forall y_3) \\
    (
        (
            \lnot
            (
                (x_2 - x_1)\cdot(y_3 - y_1) - (y_2 - y_1)\cdot(x_3 - x_1) = 0
            )
        )
        \to \\
        (\exists x_4)(\exists y_4)
        (
           (
                (x_4 - x_3)\cdot(x_2 - x_1) + (y_4 - y_3)\cdot(y_2 - y_1) = 0
            ) \& \\
            (
                (x_4 - x_2)\cdot(x_1 - x_3) + (y_4 - y_2)\cdot(y_1 - y_3) = 0
            ) \& \\
            (
                (x_4 - x_1)\cdot(x_3 - x_2) + (y_4 - y_1)\cdot(y_3 - y_2) = 0
            )
        )
    ).
\end{gather*}


Также отметим, что нет необходимости вводить в языке такие операции как вычитание, деление и возведение в степень, так как используя свойства поля и операций сравнения их можно выразить через сложение и умножение:
\begin{equation*}
    a-b=a+(-1)\cdot b; \quad \frac{a}{b} > 0 \Leftrightarrow (a > 0 \, \& \, b > 0) \lor (a < 0 \, \& \, b < 0); \quad x^2 = x \cdot x.
\end{equation*}

Получившаяся формула содержит кванторы. Если мы могли бы <<сократить>> кванторы с соответствующими переменными, то получили бы формулу логики высказываний. То есть задачу определения истинности формулы языка элементарной алгебры могли бы свести к задаче определения истинности формулы логики высказываний, которая является алгоритмически разрешимой.

\subsection{Элиминация кванторов}

Процесс <<сокращения>> кванторов принято называть элиминацией кванторов.

\begin{definition}
    \textbf{Элиминация кванторов}~--- это процесс, порождающий по заданной логической формуле, другую, эквивалентную ей бескванторную формулу, то есть свободную от вхождений кванторов.
\end{definition}

Пусть алгоритм $A$ такой, что $A\left(\left(Qx\right)\mathcal{A}\right) = \mathcal{B}$, где $\mathcal{A}$ и $\mathcal{B}$~--- бескванторные формулы языка элементарной алгебры, и формулы $(Qx)\mathcal{A}$ и $\mathcal{B}$ эквивалентны, а $Q$~--- квантор. Тогда верно следующее утверждение:
\begin{proposal}\label{algB}
    Если алгоритм $A$ существует, то существует алгоритм $B$ такой, что для любой формулы $\mathcal{A}$ языка элементарной алгебры $B\left(\mathcal{A}\right)$~--- бескванторная формула, эквивалентная $\mathcal{A}$.
\end{proposal}   
\begin{proof}
    Определим алгоритм $B$ следующим образом:
    \begin{itemize}
        \item Если $\mathcal{A}$~--- бескванторная формула, то $B\left(\mathcal{A}\right) = \mathcal{A}$;
        \item Если $\mathcal{A} = \left(Qx\right)\mathcal{B}$, то $B\left(\mathcal{A}\right) = A\left(\left(Qx\right)B\left(\mathcal{B}\right)\right)$. 
        
        Формула $B\left(\mathcal{B}\right)$~--- бескванторная по построению $B$, следовательно алгоритм $A$ можно применить к формуле $\left(\left(Qx\right)B\left(\mathcal{B}\right)\right)$. 
        
        Формула $B\left(\mathcal{B}\right)$ эквивалентна $\mathcal{B}$, следовательно, формула $\left(Qx\right)\mathcal{B}$ эквивалентна $\left(Qx\right)B\left(\mathcal{B}\right)$, а значит $\mathcal{A}$ эквивалентна $B\left(\mathcal{A}\right)$.
        
        Также заметим, что длина формулы $\mathcal{B}$ строго меньше длины формулы $\mathcal{A}$.
        \item Если $\mathcal{A}$ не удовлетворяет предыдущим условиям, то
        \begin{itemize}
            \item либо $\mathcal{A} = \lnot \mathcal{B}$, тогда $B\left(\mathcal{A}\right) = \lnot B\left(\mathcal{B}\right)$,
            \item либо $\mathcal{A} = \mathcal{B} * \mathcal{C}$, тогда $B\left(\mathcal{A}\right) = B\left(\mathcal{B}\right) * B\left(\mathcal{C}\right)$, где $* \in \left\{\lor, \&, \to\right\}$.
        \end{itemize}
        При этом длины формул $\mathcal{B}$ и $\mathcal{C}$ меньше длины формулы $\mathcal{A}$.
    \end{itemize}
    Алгоритм $B$ определен рекурсивно, при этом на каждом этапе на вход $B$ подаётся формула меньшей длины, следовательно, алгоритм $B$ является конечным, а на каждом шаге выход алгоритма~--- бескванторная эквивалентная формула.
\end{proof} 

Отметим, что доказательство не использует информацию о сигнатуре языка, следовательно данное утверждение верно и для других языков логики предикатов.

Таким образом, чтобы построить алгоритм элиминации кванторов, достаточно построить алгоритм $A$. Для языка элементарной алгебры таким алгоритмом является алгоритм Тарского.

\subsection{Алгоритм Тарского}

Прежде чем перейти к рассмотрению алгоритма, необходимо сделать ряд замечаний и предложений.

Термы в языке элементарной алгебры~--- это многочлены с рациональными коэффициентами от действительных переменных Тогда очевидно, что выражения
\begin{equation*}
    f < g, \quad f = g, \quad f > g
\end{equation*}
равносильны выражениям
\begin{equation*}
    f - g < 0, \quad f - g = 0, \quad f - g > 0 
\end{equation*}
соответственно, где $f$ и $g$~--- термы. Поэтому, не нарушая общности рассуждений, можно считать, что все атомарные формулы имеют вид:
\begin{equation*}
    f < 0, \quad f = 0, \quad f > 0.
\end{equation*}
Поэтому мы будем говорить, что нас интересует только знак многочлена: больше, меньше или равен нулю.

Очевидно, что нулевой многочлен и многочлены нулевой степени не меняют свой знак на всей области определения, их знак определяется тривиальным образом.

Известно, что многочлены от одной переменной задают непрерывные функции, следовательно, эти функции сохраняют свой знак на интервалах между корнями. Значит, чтобы уметь определять знак значения многочлена в произвольной точке, достаточно знать знак многочлена:
\begin{itemize}
    \item в его корнях, значение в которых, очевидно, равно нулю;
    \item в любой точке каждого из интервалов, на которые разбивается область определения корнями. 
\end{itemize}

Во-первых, по свойствам многочленов, множество корней ненулевого многочлена конечно. Во-вторых, конечное число точек разбивают числовую прямую на конечное число интервалов. Таким образом, необходимо знать знак многочлена лишь в \textbf{конечном} наборе точек.

Рассмотрим формулу $\mathcal{A} = (Qx)(f(x) \, \rho \, 0)$, где $Q$~--- квантор, $f(x)$~--- многочлен от одной переменной, $\rho$~--- предикат. Из наших рассуждений следует, что формуле $\mathcal{A}$ эквивалентна следующая \textbf{бескванторная} формула:
\begin{equation*}
    \mathcal{B} = 
    \begin{cases}
        \bigvee\limits_{x_0 \in X} (f(x_0) \, \rho \, 0), &\text{если $Q = \exists$} \\
        \bigwith\limits_{x_0 \in X} (f(x_0) \, \rho \, 0), &\text{если $Q = \forall$} \\
    \end{cases}
\end{equation*}
где $X$~--- конечное множество точек.

Предположим, что мы умеем находить все корни многочлена.

По теореме Ролля о нуле производной, для любой пары корней $x_1, x_2$ вещественной, непрерывной и дифференцируемой функции существует такая точка $\xi$, лежащая между $x_1, x_2$, что производная функции в точке $\xi$ обращается в ноль. Производная многочлена есть многочлен, тогда, исходя из нашего предположения, мы можем вычислить корни производной многочлена. Поэтому в качестве точек на интервале между корнями будем использовать корни производной этого многочлена.

Для интервала справа от всех корней и для интервала слева от всех корней предлагается взять любую точку, которая соответственно больше или меньше всех корней многочлена. 

Заметим, что множество корней многочлена
\begin{equation*}
    \prod\limits_{i = 1}^n f_i(x)
\end{equation*}
совпадает с объединением множеств корней многочленов $f_1(x), ... , f_n(x)$. Поэтому в качестве точек между корнями этого многочлена можно рассматривать корни многочлена
\begin{equation*}
    f_0(x) = \left(\prod\limits_{i = 1}^n f_i(x)\right)^\prime. 
\end{equation*}

Теперь нетрудно перейти от атомарных формул, к формулам общего вида. Пусть $\mathcal{A} = (Qx)(\Phi(x))$, где $\Phi(x)$~--- бескванторная формула, которая может содержать вхождения лишь переменной $x$. Тогда формула
\begin{equation*}
    \mathcal{B} = 
    \begin{cases}
        \bigvee\limits_{x_0 \in X} \Phi(x_0), &\text{если $Q = \exists$} \\
        \bigwith\limits_{x_0 \in X} \Phi(x_0), &\text{если $Q = \forall$} \\
    \end{cases}
\end{equation*}
свободна от вхождений кванторов и эквивалента формуле $\mathcal{A}$, где $X$~--- объединение множества всех корней произведения всех многочленов, входящих в $\Phi(x)$, и множества всех корней производной этого произведения.

Для того, чтобы отказаться от предположения, что мы умеем находить корни произвольного многочлена, введем ряд понятий, в том числе понятие таблица Тарского.

\subsubsection{Таблица Тарского}

Упорядочим выбранные точки $X = \left\{x_1, ... , x_s\right\}$ по возрастанию и запишем значения многочленов в этих точках в таблицу:
\begin{center}
    \begin{tabular}{ |c|c|c|c|c|c| } 
    \hline
                 & $x_1$ & $...$ & $x_j$ & $...$ & $x_s$ \\ 
    \hline
        $f_1$ & $f_1(x_1)$ & $...$ & $f_1(x_j)$ & $...$ & $f_1(x_s)$\\ 
    \hline
        $\vdots$ & $\vdots$ & $\ddots$ & $\vdots$ & $\ddots$ & $\vdots$ \\
    \hline
        $f_i$ & $f_i(x_1)$ & $...$ & $f_i(x_j)$ & $...$ & $f_i(x_s)$\\ 
    \hline
        $\vdots$ & $\vdots$ & $\ddots$ & $\vdots$ & $\ddots$ & $\vdots$ \\
    \hline
        $f_n$ & $f_n(x_1)$ & $...$ & $f_n(x_j)$ & $...$ & $f_n(x_s)$\\    
    \hline
    \end{tabular}
\end{center}

Как отмечалось ранее, нас интересуют только знаки многочленов в этих точках, поэтому в ячейках таблицы оставим лишь символ знака значения:
\begin{center}
    \begin{tabular}{ |c|c|c|c|c|c| } 
    \hline
                 & $x_1$ & $...$ & $x_j$ & $...$ & $x_s$ \\ 
    \hline
        $f_1$ & $\varepsilon_{11}$ & $...$ & $\varepsilon_{1j}$ & $...$ & $\varepsilon_{1s}$\\ 
    \hline
        $\vdots$ & $\vdots$ & $\ddots$ & $\vdots$ & $\ddots$ & $\vdots$ \\
    \hline
        $f_i$ & $\varepsilon_{i1}$ & $...$ & $\varepsilon_{ij}$ & $...$ & $\varepsilon_{is}$\\ 
    \hline
        $\vdots$ & $\vdots$ & $\ddots$ & $\vdots$ & $\ddots$ & $\vdots$ \\
    \hline
        $f_n$ & $\varepsilon_{n1}$ & $...$ & $\varepsilon_{nj}$ & $...$ & $\varepsilon_{ns}$\\    
    \hline
    \end{tabular}
\end{center}
где 
\begin{equation*}
    \varepsilon_{ij} = 
    \begin{cases}
        +, &\text{если $f_i(x_j) > 0$} \\
        0, &\text{если $f_i(x_j) = 0$} \\
        -, &\text{если $f_i(x_j) < 0$}
    \end{cases}.
\end{equation*}
Таблицы такого вида будем называть \textbf{таблицами Тарского}. 

\begin{proposal}
    Знаки $+$ и $-$ не могут стоять в двух соседних по горизонтали клетках таблицы Тарского.
\end{proposal}
\begin{proof}
    Предположим противное. Пусть многочлен $f_i$ принимает в точке $x_j$ положительное значение, а в точке $x_{j+1}$ отрицательное.
    Многочлены задают непрерывные функции, тогда, по теореме Коши о нулях непрерывной функции, существует такая точка $\xi \in (x_j, x_{j+1})$, что $f(\xi)=0$, то есть $\xi$~--- корень многочлена. Но тогда $\xi \in X$ по построению множества $X$, при этом значения $x_j$ упорядочены. Получили противоречие.
\end{proof}

\begin{proposal}\label{two zero}\cite{lect1}
    Если многочлен отличен от тождественного нуля, то в строке таблицы Тарского, соответствующей этому многочлену, в соседних по горизонтали клетках не могут стоять два символа 0.
\end{proposal}
\begin{proof}
    Предположим противное. Эти точки являются корнями многочлена, которым отмечена строка. Тогда множество $X$ должно содержать точку между этими корнями, что противоречит тому, что корни являются соседними точками в таблице Тарского.
\end{proof}

Имея таблицу Тарского, нетрудно вычислить истинностное значение формулы $\Phi(x_i)$, так как для определения истинностного значения атомарной формулы достаточно посмотреть на соответствующую клетку таблицы. При этом уже нет необходимости знать точки $x_1, ..., x_n$. А зная истинностное значение формулы, можно построить эквивалентную бескванторную:
\begin{itemize}
    \item если формула истинна~--- то можно использовать любую тождественно истинную формулу, например, $0=0$;
    \item если формула ложна~--- то можно использовать любую тождественно ложную формулу, например $0=1$.
\end{itemize}

До сих пор не обсуждалось, как искать корни многочленов. Оказывается, таблицу Тарского можно построить не находя ни одного корня, если рассмотреть системы многочленов особого вида.

\subsubsection{Насыщенная система}

\begin{definition}\cite{lect1}
    Система функций называется \textbf{полунасыщенной}, если вместе с каждой функцией, отличной от постоянной функции, она содержит и ее производную.
\end{definition}

\begin{proposal}\cite{lect1}
    Каждую конечную систему многочленов можно расширить до конечной полунасыщенной системы.
\end{proposal}
\begin{proof}
    Поместим все многочлены в очередь. Далее, пока очередь не пуста, извлекаем многочлен из очереди. Этот многочлен добавляется в множество-ответ. Если степень многочлена больше единицы, то его производная помещается в очередь. 
    
    Заметим, что на каждом шаге суммарная степень многочленов в очереди строго убывает, поэтому будет выполнено конечное число итераций алгоритма и в множестве-ответ будет конечное число многочленов.
\end{proof}

\begin{definition}\cite{lect1}
    Полунасыщенная система многочленов $p_1(x), ... , p_n(x)$ называется \textbf{насыщенной}, если вместе с каждыми двумя многочленами $p_i(x)$ и $p_j(x)$ такими, что $0 < deg(p_j(x)) \leq deg(p_i(x))$, она содержит и остаток $r(x)$ от деления $p_i(x)$ на $p_j(x)$. 
\end{definition}

\begin{proposal}\cite{lect1}
    Каждую конечную систему многочленов можно расширить до конечной насыщенной системы.
\end{proposal}
\begin{proof}
    Поместим все многочлены в очередь. Далее, пока очередь не пуста, извлекаем многочлен $f$ из очереди. Этот многочлен добавляется в множество-ответ. Если степень многочлена больше единицы, то его производная помещается в очередь. Для каждого многочлена $g$ из множества-ответ степени больше $0$ помещаем в очередь остатки от деления $f$ на $g$ и $g$ на $f$.
    
    Если на каждом шаге извлекать из очереди многочлен наибольшей степени, то количество многочленов наибольшей степени будет уменьшаться, а вместе с ним будет уменьшаться наибольшая степень многочленов, так как степень остатка от деления меньше степени обоих многочленов. Таким образом, будет выполнено лишь конечное число итераций алгоритма и в множестве-ответ будет конечное число многочленов.
\end{proof}

\begin{proposal}\label{subsystem}\cite{lect1}
    Если $p_1(x), ... , p_{n-1}(x), p_n(x)$~--- насыщенная система многочленов, и 
    \begin{equation*}
        deg(p_1(x)) \leq ... \leq deg(p_{n-1}(x)) \leq deg(p_n(x)),
    \end{equation*}
    то система $p_1(x), ... , p_{n-1}(x)$ также является насыщенной.
\end{proposal}
\begin{proof}
    Так как степень производной многочлена и степень остатка от деления меньше степени самого многочлена, то $p_n(x)$ не может являться ни производной, ни остатком от деления, поэтому после его удаления система не перестанет быть насыщенной.
\end{proof}

\begin{proposal}\label{subsystem_1}\cite{lect1}
    Если $p_1(x), ... , p_{n-1}(x), p_n(x)$~--- насыщенная система многочленов, и 
    \begin{equation*}
        deg(p_1(x)) \leq ... \leq deg(p_{n-1}(x)) \leq deg(p_n(x)),
    \end{equation*}
    то для любого натурального $m < n$ система $p_1(x), ... , p_{n-m}(x)$ также является насыщенной.
\end{proposal}
\begin{proof}
    Необходимо $m$ раз применить утверждение \ref{subsystem}.
\end{proof}

\begin{proposal}\label{min deg}
    Если $p_1(x), ... , p_n(x)$~--- насыщенная система многочленов, и 
    \begin{equation*}
        deg(p_1(x)) \leq deg(p_i(x)),\,\text{где $i = 2, 3, ... , n$},
    \end{equation*}
    то $deg(p_1(x)) < 1$.
\end{proposal}
\begin{proof}
    Если предположить противное, то с одной стороны, система должна содержать многочлен $p_1^\prime(x)$, степень которого меньше $deg(p_1(x))$, а с другой стороны, степени всех многочленов должны быть не меньше $deg(p_1(x))$, противоречие.
\end{proof}

\subsubsection{Метод построения таблицы Тарского}

Пусть $p_1(x), ... , p_{n-1}(x), p_n(x)$~--- насыщенная система многочленов, и многочлены упорядочены по не убыванию степени.

Рассмотрим подсистему из одного элемента $p_1(x)$. Согласно утверждению \ref{subsystem_1}, система $p_1(x)$ является насыщенной, тогда многочлен $p_1(x)$ представляет собой константу, так как его степень меньше единицы, согласно утверждению \ref{min deg}. В таком случае знак многочлена в любой точке совпадает со знаком этой константы. Таблица Тарского для одного многочлена имеет вид:
\begin{center}
    \begin{tabular}{ |c|c|c| } 
    \hline
        & $-\infty$ & $+\infty$ \\ 
    \hline
        $p_1$ & $\varepsilon$ & $\varepsilon$\\ 
    \hline
    \end{tabular}
\end{center}
Символами $-\infty$ и $+\infty$ обозначены точки, которые заведомо расположены левее и правее всех корней соответственно. Выбирать конкретные значения для этих точек не нужно, вместо этого предлагается считать знак предела многочлена при $x$ стремящемся к $-\infty$ и $+\infty$. Поэтому в точке $+\infty$ знак многочлена совпадает со знаком старшего коэффициента, а в точке $-\infty$ знак зависит от четности степени многочлена:
\begin{itemize}
    \item если четная, то совпадает со знаком старшего коэффициента;
    \item иначе равен знаку противоположному к знаку старшего коэффициента.
\end{itemize}
В таблице всего два столбца, поэтому верно утверждение: для каждого столбца $j$, за исключением самого правого и самого левого, в этой таблице существует ненулевой многочлен $p_i(x)$ такой, что $\varepsilon_{i, j} = 0$.

Индуктивное предположение: пусть для насыщенной системы $p_1(x), ... , p_{k-1}(x)$ уже построена таблица Тарского:
\begin{center}
    \begin{tabular}{ |c|c|c|c|c|c|c|c| } 
    \hline
                 & $-\infty$ &  & $...$ &  & $...$ &  & $+\infty$ \\ 
    \hline
        $p_1$ & $\varepsilon_{1, 1}$ & $\varepsilon_{1, 2}$ & $...$ & $\varepsilon_{1, j}$ & $...$ & $\varepsilon_{1, s-1}$ & $\varepsilon_{1, s}$ \\ 
    \hline
        $p_2$ & $\varepsilon_{2, 1}$ & $\varepsilon_{2, 2}$ & $...$ & $\varepsilon_{2, j}$ & $...$ & $\varepsilon_{2, s-1}$ & $\varepsilon_{2, s}$ \\ 
    \hline
        $\vdots$ & $\vdots$ & $\vdots$ & $\ddots$ & $\vdots$ & $\ddots$ & $\vdots$ & $\vdots$ \\
    \hline
        $p_{k-1}$ & $\varepsilon_{k-1, 1}$ & $\varepsilon_{k-1, 2}$ & $...$ & $\varepsilon_{k-1, j}$ & $...$ & $\varepsilon_{k-1, s-1}$ & $\varepsilon_{k-1, s}$\\    
    \hline
    \end{tabular}
\end{center}
И для каждого столбца $j$, за исключением самого правого и самого левого, в этой таблице существует ненулевой многочлен $p_i(x)$ такой, что $\varepsilon_{i, j} = 0$.

К этой таблице добавим строку для многочлена $p_{k}(x)$, записав знаки для крайних столбцов.
\begin{center}
    \begin{tabular}{ |c|c|c|c|c|c|c|c| } 
    \hline
        & $-\infty$ &  & $...$ &  & $...$ &  & $+\infty$ \\ 
    \hline
        $p_1$ & $\varepsilon_{1, 1}$ & $\varepsilon_{1, 2}$ & $...$ & $\varepsilon_{1, j}$ & $...$ & $\varepsilon_{1, s-1}$ & $\varepsilon_{1, s}$ \\ 
    \hline
        $p_2$ & $\varepsilon_{2, 1}$ & $\varepsilon_{2, 2}$ & $...$ & $\varepsilon_{2, j}$ & $...$ & $\varepsilon_{2, s-1}$ & $\varepsilon_{2, s}$ \\ 
    \hline
        $\vdots$ & $\vdots$ & $\vdots$ & $\ddots$ & $\vdots$ & $\ddots$ & $\vdots$ & $\vdots$ \\
    \hline
        $p_{k-1}$ & $\varepsilon_{k-1, 1}$ & $\varepsilon_{k-1, 2}$ & $...$ & $\varepsilon_{k-1, j}$ & $...$ & $\varepsilon_{k-1, s-1}$ & $\varepsilon_{k-1, s}$\\            
    \hline
        $p_{k}$ & $\varepsilon_{k, 1}$ & $ $?$ $  & $...$ & $ $?$ $ & $...$ & $ $?$ $  & $\varepsilon_{k, s}$\\    
    \hline
    \end{tabular}
\end{center}
Для каждого столбца $j$ рассмотрим многочлен $p_i(x)$ такой, что $\varepsilon_{i, j} = 0$. Этот многочлен существует и отличен от тождественного нуля в силу индуктивного предположения.
\begin{proposal}\label{остаток}
    Пусть $f(x)$ и $g(x)$~--- ненулевые многочлены. Если $g(a) = 0$, то $f(a) = r(a)$, где $r(x)$~--- остаток от деления многочлена $f(x)$ на $g(x)$.
\end{proposal}
\begin{proof}
    Многочлен $r(x)$~--- остаток от деления, тогда $f(x) = q(x)g(x) + r(x)$, подставив $a$ получим $f(a) = q(a)g(a) + r(a) = q(a)\cdot 0 + r(a) = r(a)$.
\end{proof}

Найдём $p_t(x)$~--- остаток от деления $p_k(x)$ на $p_i(x)$. Система многочленов насыщена, поэтому многочлен $p_t(x)$ уже добавлен в таблицу, тогда, применив утверждению \ref{остаток}, можем вычислить $\varepsilon_{k, j} = \varepsilon_{t, j}$. Таким образом, заполняется вся нижняя строка.

Просмотрим значения в нижней строке. Может случиться так, что в соседних клетках стоят знаки $+$ и $-$. 
\begin{center}
    \begin{tabular}{|c|c|}
        \hline
        $+$ & $-$\\
        \hline
    \end{tabular}
        \quad
    \begin{tabular}{|c|c|}
        \hline
        $-$ & $+$\\
        \hline
    \end{tabular}            
\end{center}
В таком случае необходимо добавить новый столбец между теми столбцами, в которых находятся эти клетки. Понятно, что в новом столбце нижняя клетка заполняется нулем, так как этот новый столбец заполняется для корня, существования которого следует из теоремы Коши о нулях непрерывной функции. 

Рассмотрим как заполнять новый столбец для остальных строк.
\begin{center}
    \begin{tabular}{|c|c|c|}
        \hline
        $+$ & ? & $+$\\
        \hline
    \end{tabular}
        \quad
    \begin{tabular}{|c|c|c|}
        \hline
        $+$ & ? & $0$\\
        \hline
    \end{tabular}           
\end{center}
\begin{center}
    \begin{tabular}{|c|c|c|}
        \hline
        $+$ & $+$ & $+$\\
        \hline
    \end{tabular}
        \quad
    \begin{tabular}{|c|c|c|}
        \hline
        $+$ & $+$ & $0$\\
        \hline
    \end{tabular}           
\end{center}
Тогда вместо символа $?$ ставится знак $+$, так как непрерывная функция сохраняет свой знак на интервалах между корнями.
\begin{center}
    \begin{tabular}{|c|c|c|}
        \hline
        $0$ & ? & $+$\\
        \hline
    \end{tabular}
        \quad
    \begin{tabular}{|c|c|c|}
        \hline
        $0$ & ? & $-$\\
        \hline
    \end{tabular}           
\end{center}
\begin{center}
    \begin{tabular}{|c|c|c|}
        \hline
        $0$ & $+$ & $+$\\
        \hline
    \end{tabular}
        \quad
    \begin{tabular}{|c|c|c|}
        \hline
        $0$ & $-$ & $-$\\
        \hline
    \end{tabular}           
\end{center}
В этих случаях ставится знак $+$ или $-$ соответственно.
\begin{center}
    \begin{tabular}{|c|c|c|}
        \hline
        $-$ & ? & $0$\\
        \hline
    \end{tabular}
        \quad
    \begin{tabular}{|c|c|c|}
        \hline
        $-$ & ? & $-$\\
        \hline
    \end{tabular}           
\end{center}
\begin{center}
    \begin{tabular}{|c|c|c|}
        \hline
        $-$ & $-$ & $0$\\
        \hline
    \end{tabular}
        \quad
    \begin{tabular}{|c|c|c|}
        \hline
        $-$ & $-$ & $-$\\
        \hline
    \end{tabular}           
\end{center}
В этих случаях знак $-$.
И наконец, в случае 
\begin{center}
    \begin{tabular}{|c|c|c|}
        \hline
        $0$ & ? & $0$\\
        \hline
    \end{tabular} 
\end{center}
\begin{center}
    \begin{tabular}{|c|c|c|}
        \hline
        $0$ & $0$ & $0$\\
        \hline
    \end{tabular} 
\end{center}
ставится $0$, так как эта строка точно соответствует нулевому многочлену.

А случаи
\begin{center}
    \begin{tabular}{|c|c|c|}
        \hline
        $+$ & ? & $-$\\
        \hline
    \end{tabular}
        \quad
    \begin{tabular}{|c|c|c|}
        \hline
        $-$ & ? & $+$\\
        \hline
    \end{tabular}      
\end{center}
невозможны по построению таблицы Тарского.

Таким образом, удалось построить таблицу Тарского для насыщенной системы многочленов $p_1(x), ... , p_{k-1}(x), p_k(x)$, при этом для каждого столбца найдется многочлен, на пересечении строки которого с выбранным столбцом в клетке записан символ $0$.

\subsubsection{Алгоритм для формулы вида $(Qx)\Phi(x)$}

Все готово, чтобы описать алгоритм Тарского для формулы $\mathcal{A} = (Qx)\Phi(x)$:
\begin{enumerate}
    \item Составить список всех многочленов $f_1(x), ... , f_n(x)$, входящих в $\Phi(x)$ и отличных от тождественного нуля;
    \item Добавить к этому списку многочлен 
    \begin{equation*}
        f_0(x) = \left( \prod\limits_{i = 1}^n f_i(x) \right)^\prime;
    \end{equation*}
    \item Расширить этот список до насыщенной системы $p_1(x), ... , p_m(x)$, упорядоченной по не убыванию степени;
    \item Построить таблицу Тарского; 
    \item Вычислить истинностное значение $\Phi(x)$ для каждого из столбцов таблицы;
    \item Если $Q = \exists$, то формула $\mathcal{A}$ истинна тогда и только тогда, когда хотя бы одно из вычисленных значений истинно. 
    
    Если $Q = \forall$, то формула $\mathcal{A}$ истинна тогда и только тогда, когда все вычисленные значения истинны.
\end{enumerate}

\subsubsection{Алгоритм для формулы вида $(Qx)\Phi(x, a_1, ... , a_l)$}

Оказывается, в случае, когда формула имеет вид $(Qx)\Phi(x, a_1, ... , a_l)$, нужно лишь немного модифицировать алгоритм. Во-первых, коэффициенты многочленов теперь не из $\mathbb{Q}$, а из поля частных целостного кольца $\mathbb{Q}\left[a_1, ... , a_l\right]$. Во-вторых, нельзя говорить о знаках таких коэффициентов, поэтому каждый раз, когда необходимо определить знак коэффициента, придется разбирать три случая: коэффициент меньше нуля, больше нуля или равен нулю. Поэтому будет построено дерево разбора случаев. В листьях этого дерева все знаки определены и можно построить таблицу Тарского. Если по таблице получается, что формула истинна, тогда все предположения, сделанные в ходе разбора случаев, выписываются в виде конъюнкции. Результатом же работы алгоритма будет дизъюнкция всех таких конъюнкций для каждого пути в дереве разбора.

\subsection{Пример работы алгоритма}
Рассмотрим формулу $(\forall x)(y < 0 \, \to \, x^2 > y)$ и построим эквивалентную ей бескванторную формулу с помощью алгоритма Тарского. Многочлены $y$ и $x^2 - y$ входят в данную формулу. Далее нужно выяснить все ли многочлены отличны от тождественного нуля, поэтому рассмотрим два случая:
\begin{enumerate}
    \item $y = 0$, тогда из списка исключается многочлен $y \equiv 0$ и добавляется многочлен $2x$, при этом $x^2 - y \equiv x^2$;
    \item $y < 0$ или $y > 0$, тогда в систему добавляется многочлен $2yx$.
\end{enumerate}
Переходим к насыщению системы:
\begin{enumerate}
    \item система $2x, x^2$ дополняется до $0, 2, 2x, x^2$;
    \item система $y, 2yx, x^2 - y$ дополняется до $0, y, -y, 2y, 2, 2yx, 2x, x^2 - y$.
\end{enumerate}
Построим таблицы Тарского:
\begin{enumerate}
    \item $y = 0$, система $0, 2, 2x, x^2$:
    \begin{center}
        \begin{tabular}{|c|c|c|}
            \hline
             & $-\infty$ & $+\infty$\\
            \hline
            $0$ & $0$ & $0$\\
            \hline
            $2$ & $+$ & $+$\\
            \hline
        \end{tabular} 
            \quad
        \begin{tabular}{|c|c|c|c|}
            \hline
             & $-\infty$ & & $+\infty$\\
            \hline
            $0$ & $0$ & $0$ & $0$\\
            \hline
            $2$ & $+$ & $+$ & $+$\\
            \hline
            $2x$ & $-$ & $0$ & $+$\\
            \hline
        \end{tabular}
            \quad
        \begin{tabular}{|c|c|c|c|}
            \hline
             & $-\infty$ & & $+\infty$\\
            \hline
            $0$ & $0$ & $0$ & $0$\\
            \hline
            $2$ & $+$ & $+$ & $+$\\
            \hline
            $2x$ & $-$ & $0$ & $+$\\
            \hline
            $x^2$ & $+$ & $0$ & $+$\\
            \hline
        \end{tabular}         
    \end{center}
    \item $y < 0$, система $0, y, -y, 2y, 2, 2yx, 2x, x^2 - y$:
    \begin{center}
        \begin{tabular}{|c|c|c|}
            \hline
             & $-\infty$ & $+\infty$\\
            \hline
            $0$ & $0$ & $0$\\
            \hline
            $y$ & $-$ & $-$\\
            \hline
            $-y$ & $+$ & $+$\\
            \hline
            $2y$ & $-$ & $-$\\
            \hline
            $2$ & $+$ & $+$\\
            \hline
        \end{tabular} 
            \quad
        \begin{tabular}{|c|c|c|c|}
            \hline
             & $-\infty$ & & $+\infty$\\
            \hline
            $0$ & $0$ & $0$ & $0$\\
            \hline
            $y$ & $-$ & $-$ & $-$\\
            \hline
            $-y$ & $+$ & $+$ & $+$\\
            \hline
            $2y$ & $-$ & $-$ & $-$\\
            \hline
            $2$ & $+$ & $+$ & $+$\\
            \hline
            $2yx$ & $+$ & $0$ & $-$\\
            \hline
            $2x$ & $-$ & $0$ & $+$\\
            \hline
            $x^2 - y$ & $+$ & $+$ & $+$\\
            \hline
        \end{tabular} 
            \quad    
    \end{center}
    \item $y > 0$, система $0, y, -y, 2y, 2, 2yx, 2x, x^2 - y$:
    \begin{center}
        \begin{tabular}{|c|c|c|}
            \hline
             & $-\infty$ & $+\infty$\\
            \hline
            $0$ & $0$ & $0$\\
            \hline
            $y$ & $+$ & $+$\\
            \hline
            $-y$ & $-$ & $-$\\
            \hline
            $2y$ & $+$ & $+$\\
            \hline
            $2$ & $+$ & $+$\\
            \hline
        \end{tabular} 
            \quad
        \begin{tabular}{|c|c|c|c|}
            \hline
             & $-\infty$ & & $+\infty$\\
            \hline
            $0$ & $0$ & $0$ & $0$\\
            \hline
            $y$ & $+$ & $+$ & $+$\\
            \hline
            $-y$ & $-$ & $-$ & $-$\\
            \hline
            $2y$ & $+$ & $+$ & $+$\\
            \hline
            $2$ & $+$ & $+$ & $+$\\
            \hline
            $2yx$ & $-$ & $0$ & $+$\\
            \hline
            $2x$ & $-$ & $0$ & $+$\\
            \hline
        \end{tabular} 
            \quad 
        \begin{tabular}{|c|c|c|c|c|c|}
            \hline
             & $-\infty$ & & & & $+\infty$\\
            \hline
            $0$ & $0$ & $0$ & $0$ & $0$ & $0$\\
            \hline
            $y$ & $+$ & $+$ & $+$ & $+$ & $+$\\
            \hline
            $-y$ & $-$ & $-$ & $-$ & $-$ & $-$\\
            \hline
            $2y$ & $+$ & $+$ & $+$ & $+$ & $+$\\
            \hline
            $2$ & $+$ & $+$ & $+$ & $+$ & $+$\\
            \hline
            $2yx$ & $-$ & $-$ & $0$ & $+$ & $+$\\
            \hline
            $2x$ & $-$ & $-$ & $0$ & $+$ & $+$\\
            \hline
            $x^2 - y$ & $+$ & $0$ & $-$ & $0$ & $+$\\
            \hline
        \end{tabular} 
            \quad       
    \end{center}
\end{enumerate}
Нетрудно убедиться в том, что для каждого случая, для каждого столбца формула $(y < 0 \, \to \, x^2 > y)$ истинна. В результате, на выходе алгоритма получим формулу
\begin{equation*}
    (y = 0 \lor y < 0 \lor y > 0).
\end{equation*}

\subsection{Теорема Тарского}

Таким образом, нами была доказана следующая теорема.

\begin{theorem}[Альфред Тарский]
    Для любой формулы $\mathcal{A}$ языка элементарной алгебры существует эквивалентная ей бескванторная формула этого же языка.
\end{theorem}


Алгоритм, предложенный А. Тарским в его работе \cite{Tarski}, записывался иначе и был менее понятен~--- формулы приводились к нормальным формам, явно строились эквивалентные формулы, таблицы не строились. Но и цель была не предложить <<хороший>> алгоритм, а доказать, что элементарная алгебра допускает элиминацию кванторов. В последующие годы велась работа по упрощению и усовершенствованию алгоритма, особенно в случае вхождений свободных переменных, и в результате этой работы алгоритм приобрел такой вид. Современное описание алгоритма доступнее для понимания, что упрощает его программную реализацию.