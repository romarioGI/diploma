\section*{Реферат}

Данная работа содержит \pageref{LastPage} страницы, 2 главы, 2 иллюстрации, 18 листингов исходного кода, 1 приложение, в работе использовано 8 источников.

В главе 1 вводятся понятия языка элементарной алгебры, элиминации кванторов, таблицы Тарского, насыщенной системы многочленов. Приводятся методы построения таблиц Тарского, насыщения системы многочленов. Формулируются и доказываются необходимые утверждения про насыщенные системы и таблицы Тарского. Записывается алгоритм Тарского. Формулируется теорема Тарского о языке элементарно алгебры.

В главе 2  обсуждается программная реализация алгоритма Тарского, а также архитектура и конкретные реализации её систем. Обсуждается ввод и вывод формул, лексический и синтаксический анализ формул языка элементарной алгебры. Строится грамматика этого языка. Приводятся примеры результатов работы программы. Говорится о её возможных приложениях.

\textbf{Ключевые слова}: алгоритм Тарского,
алгоритмическая разрешимость,
элиминация кванторов,
логика предикатов,
логика высказываний,
элементарная алгебра,
лексический анализ,
синтаксический анализ.