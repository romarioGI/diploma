\section*{Заключение}
\addcontentsline{toc}{section}{Заключение}

Таким образом, цели данной работы достигнуты, все поставленные задачи решены. В работе описан языка элементарной алгебры, приведены примеры задач элементарной алгебры, достаточно подробно описан алгоритм элиминации кванторов этого языка~--- алгоритм Тарского. 

Алгоритм был реализован как часть программного комплекса. Эта программа включает в себя систему ввода формул языка элементарной алгебры, библиотеки классов для представления объектов этого языка и собственно реализацию алгоритма Тарского.

В заключении хочется отметить, что алгоритм Тарского далеко не самый эффективный алгоритм, но он был первым в своем роде. Именно сконструировав этот алгоритм Альфред Тарский доказал, что элементарная алгебра допускает элиминацию кванторов, хотя до этого многие годы это считалось невозможным. 