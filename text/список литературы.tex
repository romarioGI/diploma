\begin{thebibliography}{99}
    \bibitem{lect1}
    Алгоритм Тарского. Лекция 1 // Лекториум. URL: https://www.lektorium.tv/lecture/31079 (дата обращения: 01.12.2019).

    \bibitem{lect2}
    Алгоритм Тарского. Лекция 2 // Лекториум. URL: https://www.lektorium.tv/lecture/31080 (дата обращения: 01.12.2019).

    \bibitem{Gibadulin1}
    Гибадулин Р. А. Алгоритм поиска вывода в Исчислении Высказываний и его программная реализация // Современные проблемы математики и информатики~: сборник научных трудов молодых ученых, аспирантов и студентов. / Яросл. гос. ун-т им. П. Г. Демидова.~--- Ярославль: ЯрГУ, 2019.~--- Вып. 19.~--- С. 28--37.

    \bibitem{DurnevML}
    Дурнев, В. Г. Элементы теории множеств и математической логики: учеб. пособие / Яросл. гос. ун-т. им. П. Г. Демидова, Ярославль, 2009~--- 412 с.

    \bibitem{Matiyasevich}
    Матиясевич, Ю. В. Алгоритм Тарского // Компьютерные инструменты в образовании.~--- 2008. ~--- № 6. ~--- С. 14.

    \bibitem{Sokolov}
    Соколов, В. А. Введение в теорию формальных языков : учебное пособие / Яросл. гос. ун-т им. П. Г. Демидова.~--- Ярославль : ЯрГУ, 2014.~--- 208 с.

    \bibitem{TroelsonNet}
    Троелсен, Э. Язык программирования С\# 7 и платформы .NET и .NET Core / Э. Троелсен, Ф. Джепикс; пер. с англ. и ред. Ю.Н. Артеменко. ~--- 8-е изд. ~--- М.; СПб.: Диалектика, 2020. ~--- 1328 с.

    \bibitem{Tarski}
    Tarski, A. A Decision Method for Elementary Algebra and Geometry: Prepared for Publication with the Assistance of J.C.C. McKinsey, Santa Monica, Calif.: RAND Corporation, R-109, 1951. 

\end{thebibliography}

\addcontentsline{toc}{section}{\refname}